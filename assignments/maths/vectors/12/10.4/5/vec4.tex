
\documentclass[12pt]{article}
\usepackage{graphicx}
\usepackage{amsmath}
\usepackage{mathtools}
\usepackage{gensymb}

\newcommand{\mydet}[1]{\ensuremath{\begin{vmatrix}#1\end{vmatrix}}}
\providecommand{\brak}[1]{\ensuremath{\left(#1\right)}}
\providecommand{\norm}[1]{\left\lVert#1\right\rVert}
\newcommand{\solution}{\noindent \textbf{Solution: }}
\newcommand{\myvec}[1]{\ensuremath{\begin{pmatrix}#1\end{pmatrix}}}
\let\vec\mathbf

\begin{document}
\begin{center}
\textbf\large{CHAPTER-10 \\ VECTOR ALGEBRA}

\end{center}
\section*{Excercise 10.4}

Q5.Find $\lambda \text{ and } \mu \text{ if } (2\hat{i}+6\hat{j}+27\hat{k}) \times (\hat{i}+\lambda \hat{j}+\mu \hat{k})=\vec{0}$

\solution
\begin{align}
	\text{Let } \vec{A} = \myvec{2\\6\\27} \text{ and } \vec{B} = \myvec{1\\ \lambda \\ \mu}\\
\end{align}
The cross product or vector product of $\vec{A},\vec{B}$ is defined as
\begin{align}
	\vec{A} \times \vec{B} = \myvec{\mydet{\vec{A}_{23}&\vec{B}_{23}\\\vec{A}_{31}&\vec{B}_{31}\\\vec{A}_{12}&\vec{B}_{12}}}
\end{align}
Hence
\begin{align}
	\mydet{\vec{A}_{23}&\vec{B}_{23}}&=\mydet{6&\lambda\\27&\mu}=6\mu-27\lambda\\
	\mydet{\vec{A}_{31}&\vec{B}_{31}}&=\mydet{27&\mu\\2&1}=27-2\mu\\
	\mydet{\vec{A}_{12}&\vec{B}_{12}}&=\mydet{2&1\\6&\lambda}=2\lambda-6
\end{align}
Substituting the values
\begin{align}
	\vec{A}\times\vec{B}=\myvec{6\mu-27\lambda\\27-2\mu\\2\lambda-6}
\end{align}
Now we know
\begin{align}
	\vec{A} \times \vec{B} = \vec{0}
\end{align}
So,
\begin{align}
	\myvec{6\mu-27\lambda\\27-2\mu\\2\lambda-6}=\myvec{0\\0\\0}
\end{align}
So we have three equations.
\begin{align}
	2\mu&=27\\
	2\lambda&=6\\
	6\mu-27\lambda&=0
\end{align}
The above equations can be represented in matrix form as
\begin{align}
	\myvec{2&0\\0&2\\6&-27}\myvec{\mu\\\lambda}&=\myvec{27\\6\\0}
\end{align}
The augmented matrix is given as
\begin{align}
	\myvec{2 & 0 & \vrule & 27\\0 & 2 & \vrule & 6\\ 6 & -27 & \vrule & 0}
\end{align}
Applying sequence of row operations
\begin{align}
	\xleftrightarrow{R_{3}\rightarrow R_{3}-3R_{1}}  	
	\myvec{2 & 0 & \vrule & 27\\0 & 2 & \vrule & 6\\ 0 & -27 & \vrule & -81}\\
	\xleftrightarrow{R_{3}\rightarrow R_{3}+\frac{27}{2}R_{2}}  	
	\myvec{2 & 0 & \vrule & 27\\0 & 2 & \vrule & 6\\ 0 & 0 & \vrule & 0}
\end{align}
From here we conclude that
\begin{align}
	2\mu&=27\\
	\mu&=13.5\\
	2\lambda&=6\\
	\lambda&=3
\end{align}
Hence, the values are $\lambda = 3 \text{ and } \mu = 13.5$

\end{document}
