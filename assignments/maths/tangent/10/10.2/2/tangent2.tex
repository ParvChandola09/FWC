\documentclass[12pt]{article}
\usepackage{graphicx}
\usepackage{amsmath}
\usepackage{mathtools}
\usepackage{gensymb}
\usepackage{tabularx}
\usepackage{array}
\usepackage[latin1]{inputenc}
\usepackage{fullpage}
\usepackage{color}
\usepackage{array}
\usepackage{longtable}
\usepackage{calc}
\usepackage{multirow}
\usepackage{hhline}
\usepackage{ifthen}
\usepackage{lscape}
\usepackage{float}

\newcommand{\mydet}[1]{\ensuremath{\begin{vmatrix}#1\end{vmatrix}}}
\providecommand{\brak}[1]{\ensuremath{\left(#1\right)}}
\providecommand{\norm}[1]{\left\lVert#1\right\rVert}
\providecommand{\abs}[1]{\left\vert#1\right\vert}
\newcommand{\solution}{\noindent \textbf{Solution: }}
\newcommand{\myvec}[1]{\ensuremath{\begin{pmatrix}#1\end{pmatrix}}}
\let\vec\mathbf

\def\inputGnumericTable{}

\begin{document}
\begin{center}
\textbf\large{TANGENTS AND NORMALS}

\end{center}
\section*{Excercise 10.2}
Q2.In fig \ref{fig:Fig1}, if TP and TQ are two tangents to a circle with centre O so that $\angle{POQ} = 110\degree$ then $\angle{PTQ}$ is equal to.

\solution
Let the output angle be $\phi$.
The input parameters are given as
\input{./table/table1.tex}
Any point $\vec{X}$ on the circle is given as
\begin{align}
	\vec{X} = \vec{O}+r\myvec{\cos\theta\\\sin\theta}
\end{align}
So points $\vec{P} \text{ and } \vec{Q}$ can be calculated as
\begin{align}
	\vec{P} &= \vec{O}+\myvec{\cos\theta\\\sin\theta} = \myvec{\cos\theta\\\sin\theta}\\
	\vec{Q} &= \vec{e}_1
\end{align}
For tangent $TP$
\begin{align}
	\vec{n}_1 &= \vec{P}-\vec{O}\\
	&= \myvec{\cos\theta\\\sin\theta} =  \myvec{1\\\tan\theta}\\
	\vec{m}_1 &= \myvec{1\\-\cot\theta}
\end{align}
For tangent $TQ$
\begin{align}
	\vec{n}_2 &= \vec{e}_1-\vec{O}\\
	&= \vec{e}_1\\
	\vec{m}_2 &= \vec{e}_2
\end{align}
The equation of $TP$ is given as
\begin{align}
	\vec{n}_1^\top\brak{\vec{x}-\vec{P}} &= 0\\
	\vec{n}_1^\top\brak{\vec{x}-\myvec{\cos\theta\\\sin\theta}} &= 0\\
	\label{eq:eq1}
	\myvec{\cos\theta & \sin\theta}\vec{x} &= 1
\end{align}
The equation of $TQ$ is given as
\begin{align}
	\vec{n}_2^\top\brak{\vec{x}-\vec{e}_1} &= 0\\
	\label{eq:eq2}
	\myvec{1&0}\vec{x} &= 1
\end{align}
The tangent point can be calculated by solving \eqref{eq:eq1} and \eqref{eq:eq2}
\begin{align}
	\myvec{\cos\theta&\sin\theta\\1&0}\myvec{x\\y} &= \myvec{1\\1}\\
	\label{eq:eq3}
	\implies \myvec{x\\y} &= \myvec{1\\\tan{\frac{\theta}{2}}}
\end{align}
Now, $\vec{T}=$\eqref{eq:eq3}, since it is the intersection of $TP \text{ and } TQ$. Hence, it is given as
\begin{align}
	\vec{T} = \myvec{1\\\tan{55}\degree} = \myvec{1\\1.428}	
\end{align}
The angle between two lines with slope $\vec{m}_1 \text{ and } \vec{m}_2$ is given as
\begin{align}
	\cos\phi &= \frac{\vec{m}_1^\top\vec{m}_2}{\norm{\vec{m}_1}\norm{\vec{m}_2}}\\
	&= \frac{\myvec{1&-\cot\theta}\myvec{0\\1}}{\brak{\csc\theta}\brak{1}}\\
	&= -\cos\theta\\
	\implies \cos\phi &= -\cos\theta
\end{align}
Hence,
\begin{align}
	\phi &= \cos^{-1}\brak{\cos{\brak{180\degree-\theta}}}\\
	     &= 180\degree-\theta = 70\degree
\end{align}
Hence, $\angle{PTQ} = 70\degree$. See Fig \ref{fig:Fig1}
\begin{figure}[!h]
	\begin{center} 
	    \includegraphics[width=\columnwidth]{figs/tangent2}
	\end{center}
\caption{}
\label{fig:Fig1}
\end{figure}
\newpage
Now considering the tangent point is known and verifying that the point of contacts are actually $\vec{P} \text{ and } \vec{Q}$.Given
\begin{align}
	\vec{T} = \myvec{1\\1.428}, \vec{O} = \myvec{0\\0}
\end{align}
Transforming(rotating) the tangent point using clockwise rotation matrix by $\theta = 55\degree$
\begin{align}
	\vec{T}^\prime = \myvec{\cos\theta & \sin\theta \\ -\sin\theta & \cos\theta} \myvec{1\\1.428} = \myvec{1.743\\0}
\end{align}
We know that the equation of circle is given as
\begin{align}
	\norm{\vec{x}}^2+2\vec{x}^\top \vec{u}+f=0
\end{align}
where,
\begin{align}
	\vec{u} &= -\vec{O} = -\myvec{0\\0}\\
	f &= \norm{\vec{O}}^2 - r^2 = -1
\end{align}
\begin{align}
	\vec{\Sigma} &= \brak{\vec{T}^\prime+\vec{u}}\brak{\vec{T}^\prime+\vec{u}}^\top - \brak{\norm{\vec{T}^\prime}^2 + 2\vec{u}^\top \vec{T}^\prime+f}\vec{I}\\
	&=\myvec{3.308&0 \\ 0&0} - \myvec{2.039&0 \\ 0&2.039}\\
	\label{eq:eq1}
	&=\myvec{1&0\\0&-2.039}
\end{align}
From \eqref{eq:eq1}, we can deduce Eigen pairs as follows
\begin{align}
	\lambda_1 = 1, \lambda_2 = -2.039\\
	\vec{p}_1 = \myvec{1\\0}, \vec{p}_2=\myvec{0\\1}
\end{align}
Then
\begin{align}
	\vec{n}_1 &= \myvec{1&0\\0&1}\myvec{\sqrt{\abs{\lambda_1}}\\\sqrt{\abs{\lambda_2}}} = \myvec{1\\1.427}\\
	\vec{n}_2 &= \myvec{1&0\\0&1}\myvec{\sqrt{\abs{\lambda_1}}\\-\sqrt{\abs{\lambda_2}}} = \myvec{1\\-1.427}
\end{align}
The points of contact of a tangent on a circle from an external point is given by
\begin{align}
	\vec{q}_{ij} &= \brak{\pm r\frac{\vec{n}_j}{\norm{\vec{n}_j}}-\vec{u}} \text{ i,j} = 1,2\\
	\vec{q}_{i1} &= \brak{\pm r\frac{\vec{n}_1}{\norm{\vec{n}_1}}-\vec{u}}\\
	             &= \brak{\pm\frac{1}{1.742}\myvec{1\\1.427}+\myvec{0\\0}}\\
		     &= \myvec{0.574\\0.819}\\ 		     
	\vec{q}_{i2} &= \brak{\pm r\frac{\vec{n}_2}{\norm{\vec{n}_2}}-\vec{u}}\\
	             &= \brak{\pm\frac{1}{1.742}\myvec{1\\-1.427}+\myvec{0\\0}}\\
		     &= \myvec{0.574\\-0.8191}		
\end{align}
Transforming(rotating) the coordinates back to the original using ACW rotation matrix
\begin{align}
	\label{eq:eqP}
	\vec{P} &= \myvec{\cos\theta&-\sin\theta\\\sin\theta&\cos\theta}\myvec{0.574\\0.819} = \myvec{-0.342\\0.939}\\
	\label{eq:eqQ}
	\vec{Q} &= \myvec{\cos\theta&-\sin\theta\\\sin\theta&\cos\theta}\myvec{0.574\\-0.819} = \myvec{1\\0}
\end{align}
From \eqref{eq:eqP} and \eqref{eq:eqQ} we see that the tangent point of contact are same as given before. Hence, the result is verified.

\end{document}

















